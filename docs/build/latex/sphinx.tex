%% Generated by Sphinx.
\def\sphinxdocclass{report}
\documentclass[a4paper,10pt,english,openany, oneside]{sphinxmanual}
\ifdefined\pdfpxdimen
   \let\sphinxpxdimen\pdfpxdimen\else\newdimen\sphinxpxdimen
\fi \sphinxpxdimen=.75bp\relax
\ifdefined\pdfimageresolution
    \pdfimageresolution= \numexpr \dimexpr1in\relax/\sphinxpxdimen\relax
\fi
%% let collapsible pdf bookmarks panel have high depth per default
\PassOptionsToPackage{bookmarksdepth=5}{hyperref}

\PassOptionsToPackage{booktabs}{sphinx}
\PassOptionsToPackage{colorrows}{sphinx}

\PassOptionsToPackage{warn}{textcomp}
\usepackage[utf8]{inputenc}
\ifdefined\DeclareUnicodeCharacter
% support both utf8 and utf8x syntaxes
  \ifdefined\DeclareUnicodeCharacterAsOptional
    \def\sphinxDUC#1{\DeclareUnicodeCharacter{"#1}}
  \else
    \let\sphinxDUC\DeclareUnicodeCharacter
  \fi
  \sphinxDUC{00A0}{\nobreakspace}
  \sphinxDUC{2500}{\sphinxunichar{2500}}
  \sphinxDUC{2502}{\sphinxunichar{2502}}
  \sphinxDUC{2514}{\sphinxunichar{2514}}
  \sphinxDUC{251C}{\sphinxunichar{251C}}
  \sphinxDUC{2572}{\textbackslash}
\fi
\usepackage{cmap}
\usepackage[T1]{fontenc}
\usepackage{amsmath,amssymb,amstext}
\usepackage{babel}



\usepackage{tgtermes}
\usepackage{tgheros}
\renewcommand{\ttdefault}{txtt}



\usepackage[Bjarne]{fncychap}
\usepackage{sphinx}

\fvset{fontsize=auto}
\usepackage{geometry}


% Include hyperref last.
\usepackage{hyperref}
% Fix anchor placement for figures with captions.
\usepackage{hypcap}% it must be loaded after hyperref.
% Set up styles of URL: it should be placed after hyperref.
\urlstyle{same}

\addto\captionsenglish{\renewcommand{\contentsname}{Contents:}}

\usepackage{sphinxmessages}
\setcounter{tocdepth}{1}



\title{Sphinx Documentation}
\date{Feb 01, 2024}
\release{1}
\author{lauri}
\newcommand{\sphinxlogo}{\sphinxincludegraphics{moilapp.png}\par}
\renewcommand{\releasename}{Release}
\makeindex
\begin{document}

\ifdefined\shorthandoff
  \ifnum\catcode`\=\string=\active\shorthandoff{=}\fi
  \ifnum\catcode`\"=\active\shorthandoff{"}\fi
\fi

\pagestyle{empty}
\sphinxmaketitle
\pagestyle{plain}
\sphinxtableofcontents
\pagestyle{normal}
\phantomsection\label{\detokenize{index::doc}}


\sphinxstepscope


\chapter{TEST DOCUMENTATION}
\label{\detokenize{test:test-documentation}}\label{\detokenize{test::doc}}

\section{Instalataion}
\label{\detokenize{test:instalataion}}\label{\detokenize{test:installation}}
\sphinxAtStartPar
this is how to install

\begin{sphinxVerbatim}[commandchars=\\\{\}]
\PYG{g+gp+gpVirtualEnv}{(.venv)} \PYG{g+gp}{\PYGZdl{} }pip\PYG{+w}{ }install\PYG{+w}{ }\PYG{n+nb}{test}
\end{sphinxVerbatim}


\section{Autodock}
\label{\detokenize{test:module-test}}\label{\detokenize{test:autodock}}\index{module@\spxentry{module}!test@\spxentry{test}}\index{test@\spxentry{test}!module@\spxentry{module}}\index{AnypointConfig (class in test)@\spxentry{AnypointConfig}\spxextra{class in test}}

\begin{fulllineitems}
\phantomsection\label{\detokenize{test:test.AnypointConfig}}
\pysigstartsignatures
\pysiglinewithargsret{\sphinxbfcode{\sphinxupquote{class\DUrole{w}{ }}}\sphinxcode{\sphinxupquote{test.}}\sphinxbfcode{\sphinxupquote{AnypointConfig}}}{\sphinxparam{\DUrole{n}{main\_ui}}}{}
\pysigstopsignatures
\sphinxAtStartPar
Bases: \sphinxcode{\sphinxupquote{object}}
\index{change\_properties\_mode\_1() (test.AnypointConfig method)@\spxentry{change\_properties\_mode\_1()}\spxextra{test.AnypointConfig method}}

\begin{fulllineitems}
\phantomsection\label{\detokenize{test:test.AnypointConfig.change_properties_mode_1}}
\pysigstartsignatures
\pysiglinewithargsret{\sphinxbfcode{\sphinxupquote{change\_properties\_mode\_1}}}{}{}
\pysigstopsignatures
\sphinxAtStartPar
Updates the configuration data for Mode 1 and writes it to the cached file.

\sphinxAtStartPar
This function updates the alpha, beta, and zoom parameters for Mode 1 using the
values in the UI double spin boxes. It rounds the zoom value to three decimal places
before updating the configuration data. The updated configuration data is then
written to the cached file using YAML format.
\begin{quote}\begin{description}
\sphinxlineitem{Returns}
\sphinxAtStartPar
None

\end{description}\end{quote}

\end{fulllineitems}

\index{change\_properties\_mode\_2() (test.AnypointConfig method)@\spxentry{change\_properties\_mode\_2()}\spxextra{test.AnypointConfig method}}

\begin{fulllineitems}
\phantomsection\label{\detokenize{test:test.AnypointConfig.change_properties_mode_2}}
\pysigstartsignatures
\pysiglinewithargsret{\sphinxbfcode{\sphinxupquote{change\_properties\_mode\_2}}}{}{}
\pysigstopsignatures
\sphinxAtStartPar
Updates the configuration data for Mode 2 and writes it to the cached file.

\sphinxAtStartPar
This function updates the pitch, yaw, roll, and zoom parameters for Mode 2 using the
values in the UI double spin boxes. It creates a list of tuples to store the control
names and their corresponding values, and then iterates through the list to update
the configuration data for each control. The zoom value is rounded to three decimal
places before updating the configuration data. The updated configuration data is
then written to the cached file using YAML format.
\begin{quote}\begin{description}
\sphinxlineitem{Returns}
\sphinxAtStartPar
None

\end{description}\end{quote}

\end{fulllineitems}

\index{showing\_config\_mode\_1() (test.AnypointConfig method)@\spxentry{showing\_config\_mode\_1()}\spxextra{test.AnypointConfig method}}

\begin{fulllineitems}
\phantomsection\label{\detokenize{test:test.AnypointConfig.showing_config_mode_1}}
\pysigstartsignatures
\pysiglinewithargsret{\sphinxbfcode{\sphinxupquote{showing\_config\_mode\_1}}}{}{}
\pysigstopsignatures
\sphinxAtStartPar
Reads the cached file to load the configuration data for Mode 1.

\sphinxAtStartPar
This function reads the YAML data from the cached file and loads it into the
\sphinxtitleref{self.\_\_anypoint\_config} attribute. It then sets the values of the zoom, alpha,
and beta parameters for Mode 1 in the UI double spin boxes. The function blocks
signals while updating the spin box values to prevent recursive updates. Once the
values are set, the function unblocks signals.
\begin{quote}\begin{description}
\sphinxlineitem{Returns}
\sphinxAtStartPar
None

\end{description}\end{quote}

\end{fulllineitems}

\index{showing\_config\_mode\_2() (test.AnypointConfig method)@\spxentry{showing\_config\_mode\_2()}\spxextra{test.AnypointConfig method}}

\begin{fulllineitems}
\phantomsection\label{\detokenize{test:test.AnypointConfig.showing_config_mode_2}}
\pysigstartsignatures
\pysiglinewithargsret{\sphinxbfcode{\sphinxupquote{showing\_config\_mode\_2}}}{}{}
\pysigstopsignatures
\sphinxAtStartPar
Reads the cached file to load the configuration data for Mode 2.

\sphinxAtStartPar
This function reads the YAML data from the cached file and loads it into the
\sphinxtitleref{self.\_\_anypoint\_config} attribute. It then sets the values of the pitch, yaw,
roll, and zoom parameters for Mode 2 in the UI double spin boxes. The function
blocks signals while updating the spin box values to prevent recursive updates.
Once the values are set, the function unblocks signals.
\begin{quote}\begin{description}
\sphinxlineitem{Returns}
\sphinxAtStartPar
None

\end{description}\end{quote}

\end{fulllineitems}


\end{fulllineitems}


\begin{figure}[htbp]
\centering
\capstart

\noindent\sphinxincludegraphics[scale=0.5]{{2.open-image}.png}
\caption{testing}\label{\detokenize{test:id1}}\end{figure}

\begin{sphinxadmonition}{note}{Note:}
\sphinxAtStartPar
this project is under development.
\end{sphinxadmonition}


\renewcommand{\indexname}{Python Module Index}
\begin{sphinxtheindex}
\let\bigletter\sphinxstyleindexlettergroup
\bigletter{t}
\item\relax\sphinxstyleindexentry{test}\sphinxstyleindexpageref{test:\detokenize{module-test}}
\end{sphinxtheindex}

\renewcommand{\indexname}{Index}
\printindex
\end{document}